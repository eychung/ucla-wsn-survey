%% LyX 2.1.0 created this file.  For more info, see http://www.lyx.org/.
%% Do not edit unless you really know what you are doing.
\documentclass[twocolumn,english,journal]{IEEEtran}
\usepackage[T1]{fontenc}
\usepackage{color}
\definecolor{page_backgroundcolor}{rgb}{1, 1, 1}
\pagecolor{page_backgroundcolor}
\usepackage{babel}
\usepackage{graphicx}
\usepackage[unicode=true,
 bookmarks=true,bookmarksnumbered=true,bookmarksopen=true,bookmarksopenlevel=1,
 breaklinks=false,pdfborder={0 0 0},backref=false,colorlinks=false]
 {hyperref}
\hypersetup{pdftitle={Your Title},
 pdfauthor={Your Name},
 pdfpagelayout=OneColumn, pdfnewwindow=true, pdfstartview=XYZ, plainpages=false}

\makeatletter
%%%%%%%%%%%%%%%%%%%%%%%%%%%%%% Textclass specific LaTeX commands.
\newenvironment{lyxcode}
{\par\begin{list}{}{
\setlength{\rightmargin}{\leftmargin}
\setlength{\listparindent}{0pt}% needed for AMS classes
\raggedright
\setlength{\itemsep}{0pt}
\setlength{\parsep}{0pt}
\normalfont\ttfamily}%
 \item[]}
{\end{list}}

%%%%%%%%%%%%%%%%%%%%%%%%%%%%%% User specified LaTeX commands.
% for subfigures/subtables
\usepackage[caption=false,font=footnotesize]{subfig}

\makeatother

\begin{document}

\title{Using Wireless Sensor Networks to Save Lives from Pollution Terrorism}


\author{Eric Chung, Gabriel Alsheikh}
\maketitle
\begin{abstract}
Today, pollution is a significant consideration in public health due
to the immense burning of fossil fuels and other chemicals for industries,
power generation, vehicles and a multitude of other uses. This is
especially the case in urbanized areas where people and sources of
pollutants are densely concentrated. In this paper, we cover several
works on wireless sensor networks and pollution monitoring. We focus
on a prominent work on pollution monitoring that involves a wireless
sensor network with mobile and static node that is able to process
and deliver information in real-time. We also cover other relevant
and vital papers established in this area of technology, including
different architectures and different deployment schemes (eg. public
transportation, personal mobile devices, static nodes). We close our
survey with a discussion of the state-of-the-art technologies in pollution
monitoring wireless sensor networks, including both solved and existing
problems, and our expectations of its future direction based on current
developments and opportunities.\end{abstract}
\begin{IEEEkeywords}
Wireless Sensor Networks, Air Pollution
\end{IEEEkeywords}

\section{Introduction}

Industrialization, increased energy use, traffic, and growing urban
centers, among other factors have caused air pollution to be a serious
issue in many countries not just for the environment, but public health.
In the United States, 79\% of Americans live in urban areas \cite{us_census_bureau}
and are subjected to a dramatically increased degree of air pollution
than would exist outside urban and industrial centers. The US Environmental
Protection Agency the six most common air pollutants are particulate
matter, carbon monoxide, sulfur oxides, nitrogen oxides, ozone, and
lead \cite{us_environmental_protection_agency}. In the United States,
COPD, asthma, and other respiratory diseases plague millions of people.
COPD is the third leading cause of death in the country \cite{american_lung_association.a}
and 26 million Americans suffer from asthma. Air pollution is prominent
in worsening symptoms and effects of these fatal diseases \cite{american_lung_association.b}.
Studies have demonstrated results showing how destructive the effects
of air pollution can be, particularly on those afflicted by respiratory
illnesses in polluted metropolitan areas like Los Angeles \cite{usc_news}.

Wireless sensor networks (WSNs) enable gathering data wirelessly from
a large number of nodes in almost real-time. In a world of air pollution
harming the environment and human health, there does not exist a strategy
to thoroughly monitor air pollution over a wide area and in different
urban settings. To contrast, current methods of pollution monitoring
via expensive equipment at fixed locations or laboratory analysis
are inefficient in providing an accurate image of air pollution across
a broad range of environments even within a single city. The granularity
is extremely coarse and available data points are limited and data
collection times are spaced out. These methods\textquoteright{} primary
disadvantages lie in the facts that they cannot usually be achieved
in real-time and they involve extrapolating models that give a very
imprecise, unreliable of pollution outside of the immediate area of
where pollution data was collected. In addition, this data, as limited
as it is, it also limited in its accessibility. However, WSNs allow
for data at many points within a given area, real-time data collection
and dissemination, and an effective scheme to monitor air pollution
and formulate an accurate image of its proliferation and concentration.
WSNs provide a high-granularity, real-time system that can provide
high quantities of data in a dense area, which is necessary for effective
awareness of air pollutants and their effect on health issues.

In this paper, we present a survey on different strategies in which
WSNs are used for pollution monitoring. We discuss the various advantages
and challenges involved in a wireless sensor network. This includes
the architecture of wireless sensor networks and how it can produce
a successful solution for effective pollution monitoring. We highlight
various deployment schemes to demonstrate the effectiveness and feasibility
of pollution monitoring WSNs. Our focus will be on analyzing the different
solutions and strategies that have been proposed, and additionally
on evaluating their benefits in solving the pollution monitoring problem.
Critical emphasis will be placed on mobile, portable solutions that
can be employed on public transportation or individual users, as these
appear to be the most promising strategies. 


\section{Key Paper}

The paper \textquotedblleft Air Pollution Monitoring and Mining Based
on Sensor Grid in London\textquotedblright{} \cite{y._ma} provides
a very comprehensive and detailed work on a wireless sensor network
for air pollution monitoring. The goal is to develop low-cost, pervasive
sensor networks to gather real-time air pollution data from roadways
over a large area. This work is motivated by the high levels of air
pollution as a result of road traffic, which contributes 97\% of CO
and 75\% of NOX emissions in the city, along with lead, ozone, benzene,
and particulate matter. All of these have a seriously detrimental
effect on the environment and human health. In addition, existing
monitoring stations are often only at pollution hotspots, may be several
kilometers apart from each other, and the data is processed offline,
which limit such a system\textquoteright s effectiveness in thoroughly
monitoring air pollution. The data gaps in pollution monitoring have
produced three major barriers: the lack of validation of traffic and
pollution models, inability to determine individual exposure to pollutants,
and a lack of integrated traffic and environmental control. These
concerns can be addressed by generating new types of useful data and
more data at both the temporal and spatial levels. The work proposes
the Mobile Discovery Net (MoDisNet), a low-cost mobile monitoring
system to develop a grid structure to improve monitoring of harmful
environmental gases.

The MoDisNet air pollution monitoring system has an infrastructure
based on the use of vehicles as a platform for environment sensors
along with static sensors placed at roadside locations in a grid setting.
More specifically, the system has a two-layer network architecture
with one layer being the Mobile Sensor Nodes (MSN) and the other being
the Static Sensor Nodes (SSN). The data collected by MSNs is sent
to SSNs for data processing, storage, receiving, and uploading. This
is shown in the figure \cite{y._ma}.

\begin{center}
\begin{figure}[tbh]
\begin{lyxcode}
\includegraphics[width=1\columnwidth]{\string"Figure 1\string".eps}
\end{lyxcode}
\centering{}\protect\caption{MoDisNet network architecture}
\end{figure}

\par\end{center}

The GUSTO (Generic Ultraviolet Sensor Technologies and Observations)
sensors designed and used in this system are capable of simultaneous
detection of multiple pollutants, real-time data collection, low cost
per unit, data robustness, and high accuracy of pollution concentrations.
When a vehicle passes a sensor, the detected pollutants (SO2, NOx,
O3, etc.) exhausted from the vehicle are detected by the GUSTO sensor
and the data is processed to determine the concentration. The nodes
in the wireless sensor network communicate each other as a P2P (peer-to-peer)
network. Due to the gigabit magnitude of data transferred per sensor
each day, the sensors must store and communication the information,
which is filtered and processed before being stored in a backend repository.
The GUSTO sensors connect to the MoDisNet grid through Sensor Gateways
according to different wireless protocols (WiFi, Zigbee \cite{d._a._jadhav},
etc.). The sensors send their data to the gateways across the network,
in a multi-hop manner across sensors if necessary. Meanwhile, the
SGs are responsible for handling the data between the sensors and
the backend data warehouse and for load balancing of the data traffic.
The backend warehouse stores the data from the sensors along with
other data such as traffic, weather, and health information, which
can be used by the grid to create air pollution and traffic models.
In addition, it can provide real-time information to end users about
the current environmental conditions. Figure 2 \cite{y._ma} shows
the whole MoDisNet sensor network architecture.

\begin{figure}[t]
\includegraphics[width=1\columnwidth]{\string"Figure 2\string".eps}

\begin{centering}
\protect\caption{MoDisNet sensor grid architecture}

\par\end{centering}

\end{figure}


Data mining techniques are critical to the research and study of the
relationship between the environment and road transportation. Often,
end users will have to query data from the backend data storage to
understand the current environmental setting. In this area, centralized
data mining is implemented in the backend with the purpose of discovering
long-term pollution and traffic patterns and finding the relationship
between specific traffic and pollution events. P2P-based distributed
data mining is implemented in the sensors, as it depends on data collected
and stored by each sensor in real-time. By mining this real-time data
straight from the sensors, a quick analysis can be made on the pollution
concentration and as a result, on the traffic situation as well. However,
the results may be simple and not as thorough and accurate as the
centralized mining results. There is also integrated data mining,
where the distributed mining results match a pattern retrieved from
the centralized mining leading to the assumption that some event will
happen, or there is a new result that leads to an updated, more accurate
model in the centralized mining area. Data mining in sensor networks
faces challenges such as the resource constraints on the sensors (battery,
communication bandwidth, etc.), the mobility of sensor nodes which
adds complexity to data collection and analysis, and sensor data is
transferred in real-time over the network which cannot make use of
traditional backend schemes. Distributed data mining can address these
problems by using distributed resources in an optimal manner, with
focuses on intelligent data collection schemes to reduce volume, optimized
node organization schemes to increase homogeneity of the data, and
efficient mining algorithms to reduce computing complexity. 

MoDisNet has a simulation platform which allows it to give an evaluation
of the hardware/software design, with a focus on simulation the wireless
communication processing, the sensor units,upper and lower layer algorithms
and programs, and the cooperation between the MSN and SSN types of
sensors. The simulation can be visualized to demonstrate the effectiveness
of the configuration of the system and performance of the routing
protocols and data transfer. The authors simulate an 18 sensor node
(12 SSNs and 6 MSNs) which can send data bidirectionally to sensors
within range. Air pollution data for four pollutants is taken at every
minute in an urban setting, showing that the system can receive and
send messages from node to node with minimal collision and packets
lost. Another evaluation demonstrates the ability of MoDisNet in a
real-time pollution data monitoring scenario, based on actual data
collected from a sensor grid. The evaluation is not only able to determine
where there is high pollution, but which gases contribute to the pollution
as well, which can be organized into pollution clouds. Figure 3 \cite{y._ma}
shows the map with the pollution clouds at different times of the
day, while Table 1 \cite{y._ma} exhibits what each of the clouds
mean. For example, the results show that at 9:00, the pollution is
greatest near schools (circles) and main roads, at 15:30 by the schools
and factory (square), and at 17:00 by the hospitals (ovals), factory,
and main roads. 

\noindent 
\begin{figure*}[t]
\begin{centering}
\includegraphics{\string"Figure 3\string".eps}
\par\end{centering}

\begin{centering}
\protect\caption{Pattern recognition for high air pollution level areas}

\par\end{centering}

\end{figure*}


\begin{table}[tbh]


\protect\caption{Pollution patterns}


\includegraphics[width=1\columnwidth]{\string"Table 1\string".eps}
\end{table}



\section{Additional Papers}

In this section, we will address specific additions that the key paper
does not cover but are an area of interest as research can be furthered
in these directions to improve WSN systems as a whole. Firstly, MoDisNet
does not focus on different deployment mediums. There are numerous
types as will be mentioned later in this survey, but we focus on a
mobile phones for the next part. However, there are pros and cons
from deviating from the traditional state of the art system. Secondly,
we evaluate a newer routing protocol presented in another paper that
outperforms current WSNs in respect to energy efficiency. Lastly,
we examine an improved clustering algorithm that also focuses on reducing
energy consumption.


\subsection{Deployment via Mobile Phones}

In \textquotedblleft Wireless Sensor Network Deployment in Mobile
Phones Assisted Environment\textquotedblright{} \cite{z._ruan}, the
idea behind using mobile phones as a deployment medium sheds a new
light from the current state of the art. The benefits of using mobile
phones include reducing the number of WSN units deployed, which also
corresponds to the decrease costs in equipment and maintenance of
the units. Rather than replacement of WSNs, the paper suggests that
mobile phones can serve as a complement to existing WSN units and
integration through their framework seamlessly. In order to prevent
quality loss of areas not covered, the paper proposes a cost-effective
sensor deployment algorithm that covers the necessary areas to meet
application needs. This greedy algorithm determines where a mobile
sensor should be placed on the grid based on the missing coverage
probability and sensing quality. It focuses on areas where coverage
is least effective and iterates through the list of grid points until
all points are covered or an upper bound of sensors is met. In order
to evaluation the effectiveness of the system, the paper introduces
a sensing quality evaluation model and a mobility prediction model.

\begin{figure}[t]
\includegraphics[width=1\columnwidth]{\string"Figure 4\string".eps}

\begin{centering}
\protect\caption{System architecture}

\par\end{centering}

\end{figure}


Building off the model presented by the current state of the art,
this paper reflects the network structure presented in the key paper
except where mobile phones replace the need for sensors in certain
areas as shown in Figure 4 \cite{m._razzaque}. By reducing the overall
cost of the system, the use of mobile phones introduce an array of
sensor quality issues due to trustworthiness of users and data quality
on phones for instance.


\subsection{Improvement on Routing Protocols}

A major networking issue in all WSNs is the unexpected downtime for
a node, so WSN systems rely on an underlying routing protocols tailored
to delay-sensitive networks. An interesting approach to load balance
the system while providing fault tolerance is to use the knowledge
of neighboring nodes to determine the next course of action. In \textquotedblleft QoS-aware
distributed adaptive cooperative routing in wireless sensor networks\textquotedblright{}
\cite{m._razzaque}, the authors propose a distributed adaptive cooperative
routing protocol (DACR) to answer the issue of cooperative routing
in WSNs. The main idea is to leverage cooperative communication and
exploit knowledge of delay- and energy-aware routes in order to make
trade-offs between reliability and delay. As each node in the system
is not reliable for whatever reason, the underlying network protocol
must be able to continue the operations of the whole system and transmit
the necessary sensor data despite a node failure. In order to maintain
a reliable system, DACR effectively determines, at each hop, which
transmission mode, direct or relay, to pick in order for the system
to continue running smoothly.

Compared to state of the art protocols in current WSN systems, DACR
operates better in energy-distribution and load balancing using its
network traffic prediction approach. DACR, like AODV, uses proactive
relay selection, which is shown to improve QoS performance compared
to using reactive selection. The paper highlights how the use of cooperative
relay nodes reduce computation cost, leading to higher energy cost
performance. In detail, the number of route reconstructions and packet
retransmissions is minimized by the dynamic decisions made at each
node.


\subsection{Zone Partitioning in WSNs}

The paper, \textquotedblleft Power-Efficient Zoning Clustering Algorithm
for Wireless Sensor Networks\textquotedblright{} \cite{f.-e._bai},
presents a clustering routing algorithm, called PEZCA, to improve
on the energy-efficiency of WSNs. It works on the principles of Low-Energy
Adaptive Clustering Hierarchy (LEACH) and Power-Efficient Gathering
in Sensor Information System (PEGASIS) and builds off on current state
of the art routing protocols. LEACH randomly selects cluster head
nodes and rotates the responsibility in order to allow for even energy
dissipation. As an improvement of LEACH, PEGASIS allows for chaining
and data aggregation. PEZCA, which takes ideas from both, can be thought
of as partitioning the nodes into fan-shaped regions. Clusters of
nodes that lie closer to the base station are smaller, whereas those
farther away are larger. The idea is to reduce inter-cluster relay
traffic, which in turn lowers energy consumption.

Reduced energy consumption using PEZCA compares better against current
state of the art clustering algorithms, which share typically longer
chains among cluster heads. By allowing different cluster sizes and
forming a fan-shaped configuration as shown in Figure 5 \cite{f.-e._bai}
originating from the base station, energy consumption is balanced
amongst the system.

\begin{figure}[tbh]
\includegraphics[width=1\columnwidth]{\string"Figure 5\string".eps}

\protect\caption{Overview of PEZCA}


\end{figure}



\section{State of the Art}

Current WSN architectures for air pollution monitoring systems have
static or mobile energy-efficient sensors that are connected to a
backend infrastructure. These nodes upload their collected sensor
information and operate only when needed in order to save energy.
Generally, there are three configurations for when a WSN node should
upload: time-driven, event-driven, and query-driven. Time-driven uploads
occur over a set interval. Whereas, event-driven uploads happen in
response as an alert to a particular event, and query-driven is when
the sink requests for the data. Although a given node can only record
pollution information for its own respective zone, together these
nodes can collaborate to generate a map that covers a wide scale area.
Applications then retrieve a summary of the processed data from the
backend servers for the users.

Within a WSN unit, which can be mounted on a sensing station as shown
in Figure 6 \cite{g._barrenetxea}, the core components include a
power source, a radio receiver, and a sensor board. In most cases,
there are usually two batteries attached to the sensor unit.

\begin{figure}[tbh]
\includegraphics{\string"Figure 6\string".eps}

\protect\caption{Sensing station}


\end{figure}



\subsection{Problems Unsolved}

The most common issues that arise with any WSN system and remain unsolved
include communication bandwidth and scalability. 


\subsubsection{Communication Bandwidth}

Communication bandwidth becomes an issue if the system prefers quality
data where a high frequency of sensor information updates need to
be perpetually sent between WSN nodes and the servers or if a sink
node has too many incoming connections. As a result, a build up of
data packets at network bottlenecks will slow down system operations
and provide delayed analysis for the end users. This challenge was
deal with by the use of various routing algorithms \cite{m._razzaque}
that have built in congestion control mechanisms that alleviate hot
spot traffic by rerouting traffic to less impacted data paths. 


\subsubsection{Scalability}

As WSNs can grow large in size, the challenge of scalability arises.
Some solutions that are offered include clustering algorithms such
that geographically close nodes form a cluster and routing algorithms
that focus on load balancing the system. In order to efficiently communicate
data to the sink, cluster heads are identified within the network
structure to propagate sensor information collected by the nodes that
they are responsible for. If a cluster head goes down, the system
can even elect a new replacement cluster head to represent the group
of nodes. There is currently a large list of clustering algorithms
\cite{v._kumar}, which differ on how clusters are formed, how cluster
heads are selected, and how nodes propagate data within clusters.
Depending on the application needs, one algorithm may be preferenced
over another.


\subsection{Challenges}

Many challenges still remain problematic for the state of the art
architecture for WSNs, in which the main issues are power, robustness,
and storage space. Although the state of the art architecture tries
to approach these problems through alternative means, these challenges
remain fundamental in WSN systems and solid solutions for the underlying
problems are still sought after. 


\subsubsection{Power}

Energy is the fundamental constraint in any WSN system. A node uses
energy every time it collects data and transmits it, so unless the
WSN unit is connected to a reliable battery source, the device should
only be powered on when it needs to. In order to provide quality sensor
data, the state of the art WSN nodes must transmit frequently or perform
a tradeoff with power. If a static node runs out of power, the battery
must be physically replaced in order for the unit to remain operational.
As for mobile nodes, the battery should not interfere with other operations
of the integrated device. Attempts have been made to solve the efficient-energy
coverage (EEC) problem by designing better routing, clustering, and
scheduling \cite{j-w._lee} algorithms.


\subsubsection{Robustness}

Because WSN nodes can fail without reason, the system needs to properly
accommodate for these occurrences. If a node that is responsible for
relaying data from a group of other nodes fails, these other nodes
need to propagate their data through other network paths. The node
would then need to be physically replaced by a functional unit in
order to continue recording information for that area. 


\subsubsection{Storage Space}

Because current state of the art nodes are resource-limited devices,
storing large amounts of data becomes an issue. Due to power constraints,
uploading sensor information frequently is inadvisable, so a unit
must be able to maximize its storage per push to sink nodes. An alternative
approach to handle the storage space issue is to distribute the data
to local nodes through a cooperative storage mechanism \cite{a._awad}.


\section{Future Direction}

As far as research goes on WSNs for pollution applications, improvements
in deployment medium, network protocols, data processing, and hardware
technologies contribute to developing a better system for monitoring
the air quality in our environment.


\subsection{Deployment Mediums}

Moving away from the traditional deployment of static nodes, mobilizing
the nodes becomes a step forward for the research community. In specific,
the idea of integrating sensors on bikes \cite{k._g._sadik}, on buses
\cite{s._devarakonda}, in cars \cite{b._hull}, \cite{w._tsujita},
and with phones \cite{d._hasenfratz} is becoming a greater interest
as it tackles the problem of monitoring air pollution in another light.
By personalizing the technology, people \cite{a._t._campbell} are
able to become more aware of their health as a result of knowing the
air quality around them. However, this focus does not take away the
important operations current pollution monitoring systems perform.
Valuable sensor information will continue to be sent to backend servers
for data analysis.


\subsection{Network Protocols}

For WSNs to be efficient, new network protocols must be tailor made
to fit the specific needs and limitations of a system. The location
of nodes and their estimated time spent online influence how and where
data packets should be processed, which becomes a study into routing
algorithms. In specific, researchers have begun looking into new clustering
\cite{a._del_coso} and scheduling algorithms for how groups of related
nodes should be identified and how often nodes should propagate their
data to the next node.


\subsection{Data Processing}

In order to manage the large amount of incoming data to sink servers,
the system must be able to efficiently process all this sensor information.
Because storage space within WSN units is an issue, the unit should
identify critical sensor inputs and filter out data impurities \cite{k._k._khedo}.
Future research work should focus on techniques for identifying outliers
versus actual change in air quality in order to improve the accuracy
in data processing. On a similar note, the unit should only activate
to capture as often as the application sees fit. Therefore, improvements
in scheduling algorithms will help manage how much outgoing data will
be processed. 


\subsection{Hardware Technologies}

The most recent major improvement in hardware technology for our study
was the use of MEMS (micro electromagnetic systems). Due to their
robustness and small size, they can produce a two-fold advantage over
existing gas sensors: more accurate air pollutant measurements and
a much smaller size. Due to this, it makes further miniaturization
of these pollution sensors possible. This is particularly helpful
for personal, mobile, and non-obstructive sensors that can even be
toted with end users without creating any impediment. Overall, MEMS-based
gas sensors can revolutionize pollution monitoring wireless sensor
networks. 


\subsection{Public Policy}

As pollution monitoring becomes increasingly more effectives, thorough,
and convincing, it will become more difficult to ignore the harmful
effects of air pollution, not only at the social level, but at the
policymaking level as well. If public policymakers are presented with
irrefutable and large-scale evidence of the correlation between air
pollution and respiratory disease, then it will become more likely
for them to support policies that would lead to technologies, systems,
and practices that could improve air quality and reduces its harm
on people at risk of or suffering from respiratory disease. In this
manner, there will be a much stronger push for greener approaches
in regards to air and environmental quality. 


\section{Conclusions}

In the modern era, pollution is becoming extremely harmful to human
health. Despite its importance, we lacked a reasonably effective technology
to monitor air quality until the introduction of pollution monitoring
wireless sensor networks. Our survey discussed prominent works in
this emerging technology. We place particular detail on the work describing
the Mobile Discovery Net, which employs mobile and static nodes in
a sensor network to collect air quality data and process it. We also
covered other works on pollution monitoring in order to exhibit other
schemes and strategies that exist in this area, such as deployment
via mobile phones. In addition, we discussed the state-of-the-art
in this field, with its various innovations and existing challenges.
We also wrote on the future direction for this exciting technology
in terms of usage and future innovation. Overall, the future is bright
for this technology and hopefully it will continue to broaden our
understanding of air pollution and reduce the effects of poor air
quality on human health. 

\bibliographystyle{IEEEtran}
\bibliography{references}

\end{document}
